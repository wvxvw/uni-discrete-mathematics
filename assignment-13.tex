% Created 2015-04-26 Sun 00:47
\documentclass[11pt]{article}
\usepackage[utf8]{inputenc}
\usepackage[T1]{fontenc}
\usepackage{fixltx2e}
\usepackage{graphicx}
\usepackage{longtable}
\usepackage{float}
\usepackage{wrapfig}
\usepackage{rotating}
\usepackage[normalem]{ulem}
\usepackage{amsmath}
\usepackage{textcomp}
\usepackage{marvosym}
\usepackage{wasysym}
\usepackage{amssymb}
\usepackage{hyperref}
\tolerance=1000
\usepackage[utf8]{inputenc}
\usepackage[usenames,dvipsnames]{color}
\usepackage[backend=bibtex, style=numeric]{biblatex}
\usepackage{commath}
\usepackage{eufrak}
\usepackage{tikz}
\usetikzlibrary{shapes,backgrounds}
\usepackage{marginnote}
\usepackage{listings}
\usepackage{color}
\usepackage{enumerate}
\hypersetup{urlcolor=blue}
\hypersetup{colorlinks,urlcolor=blue}
\addbibresource{bibliography.bib}
\setlength{\parskip}{16pt plus 2pt minus 2pt}
\definecolor{codebg}{rgb}{0.96,0.99,0.8}
\definecolor{codestr}{rgb}{0.46,0.09,0.2}
\author{Oleg Sivokon}
\date{\textit{<2015-04-05 Sun>}}
\title{Assignment 13, Discrete Mathematics}
\hypersetup{
  pdfkeywords={Discrete Mathematics, Assignment, Relations},
  pdfsubject={Third asssignment in the course Discrete Mathematics},
  pdfcreator={Emacs 25.0.50.1 (Org mode 8.2.2)}}
\begin{document}

\maketitle
\tableofcontents


\lstset{ %
  backgroundcolor=\color{codebg},
  basicstyle=\ttfamily\scriptsize,
  breakatwhitespace=false,         % sets if automatic breaks should only happen at whitespace
  breaklines=false,
  captionpos=b,                    % sets the caption-position to bottom
  commentstyle=\color{mygreen},    % comment style
  framexleftmargin=10pt,
  xleftmargin=10pt,
  framerule=0pt,
  frame=tb,                        % adds a frame around the code
  keepspaces=true,                 % keeps spaces in text, useful for keeping indentation of code (possibly needs columns=flexible)
  keywordstyle=\color{blue},       % keyword style
  showspaces=false,                % show spaces everywhere adding particular underscores; it overrides 'showstringspaces'
  showstringspaces=false,          % underline spaces within strings only
  showtabs=false,                  % show tabs within strings adding particular underscores
  stringstyle=\color{codestr},     % string literal style
  tabsize=2,                       % sets default tabsize to 2 spaces
}

\clearpage

\section{Problems}
\label{sec-1}

\subsection{Problem 1}
\label{sec-1-1}
Demonstrate five distinct stes: $A$, $B$, $A \cup B$, $A \oplus B$, $A \setminus B$,
all of which have the same cardinality.

\subsubsection{Answer 1}
\label{sec-1-1-1}
There's a popular programming puzzle called ``FizzBuzz''.  The taks is to
print natural numbers according to the rule: if a number is divizible by
three, the program prints \texttt{fizz}, if the number is divisible by five the
program prints \texttt{buzz} (when both condition holds, the program thus prints
\texttt{fizzbuzz}, hence the name).  Otherwise the program prints the number itself.

It is easy to see that the description above perfectly matches the requirement:
\begin{equation*}
  \begin{aligned}
    A             &= \{x \in \mathbb{N} \; | \; x \bmod 3 = 0\} \\
    B             &= \{x \in \mathbb{N} \; | \; x \bmod 5 = 0\} \\
    A \cup B      &= \{x \in \mathbb{N} \; | \; x \bmod 3 = 0 \lor x \bmod 5 = 0\} \\
    A \oplus B    &= \{x \in \mathbb{N} \; | \; (x \bmod 3 = 0 \lor x \bmod 5 = 0)
                      \land \lnot (x \bmod 15 = 0)\} \\
    A \setminus B &= \{x \in \mathbb{N} \; | \; x \bmod 3 = 0 \land \lnot (x \bmod 5 = 0)\}.
  \end{aligned}
\end{equation*}

The cardinality of all sets is $\aleph_0$ and neither one of them is equal to any
other set.
\subsection{Problem 2}
\label{sec-1-2}
\begin{enumerate}
\item Let $n$ be a natural number. Show that the set of all subsets of $\mathbb{N}$ of
cardinality exactly $n$ is a countable set.
\item Show that the set of all finite subsets of $\mathbb{N}$ is countable.
\item Show that the set of all infinite subsets of $\mathbb{N}$ is not countable.
\item Find the cardinality of the set given in (3).
\item $\aleph_0 = \abs{\{X \in P(\mathbb{N}) | \abs{X} = n\}}$ is the restatement of (1).
Write a similar statement for (2) and (4).
\end{enumerate}

\subsubsection{Answer 2}
\label{sec-1-2-1}

\subsubsection{Answer 3}
\label{sec-1-2-2}

\subsubsection{Answer 4}
\label{sec-1-2-3}

\subsubsection{Answer 5}
\label{sec-1-2-4}

\subsubsection{Answer 6}
\label{sec-1-2-5}
\subsection{Problem 3}
\label{sec-1-3}
Find the error in the following:

\begin{quote}
We shall define simmetric set difference as follows: let $k$ and $m$ be cardinalities,
not necessarily distinct.  Define $k \oplus m$ to mean $A$ and $B$ sets for which
$\abs{A} = k$ and $\abs{B} = m$, then the cardinality of $A \oplus B$ is the value
of $k \oplus m$.
\end{quote}

\subsubsection{Answer 7}
\label{sec-1-3-1}
Since we are trying to define a \emph{binary operation}, we must make sure that the
result (or value) of this operation is uniquely defined.  But it is easy to see
it is not the case, since if $\abs{A \cap B} \neq \abs{A' \cap B'}$
while at the same time $\abs{A} = \abs{A'}$ and $\abs{B} = \abs{B'}$, we will get
that $k \oplus m \neq k \oplus m$.  Thus our attempt at defining a ring-sum
operation will fail.
\subsection{Problem 4}
\label{sec-1-4}
\begin{enumerate}
\item Let $k_1, k_2, m_1, m_2$ be cardinalities. Prove that if $k_1 \leq k_2$ and
      $m_1 \leq m_2$ then $k_1 \times m_1 \leq k_2 \times m_2$.
\item Prove that $\mathfrak{c} \times \aleph_0 = \mathfrak{c}$.
\item Prove that $\mathfrak{c}^{\mathfrak{c}} = 2^{\mathfrak{c}}$.
\end{enumerate}

\subsubsection{Answer 8}
\label{sec-1-4-1}
Recall that the product of cardinalities is defined to be the cardinality of
cartesian product.  Also, recall that $\abs{A} \leq \abs{B}$, implies having
an injective function from $A$ to $B$.  Now we can combine these two facts
to build the proof.  Suppose for contradiction that it was possible that
$k_1 \times m_1 > k_2 \times m_2$.  This would mean that for some sets $A$
and $B$ the cartesian product with cardinality $k_1 \times m_1$ defined by
$A \times B$ contains a pair $(a, b)$ with a property that for the chosen
injective function from $A$ to $A'$ $a$ is not the source of any element in
$A'$. But we are given that there is an injective function from $A'$ to $A$
(because $k_1 \leq k_2$).  The argument for $b$ is identical.  Recall that
the function must be defined for each element in its domain, but this stands
in contradiction to the earlier claim that $a$ (or $b$) is not in the domain
of this function.  Since this is not possible, the proof is complete.
\subsubsection{Answer 9}
\label{sec-1-4-2}
To prove $\mathfrak{c} \times \aleph_0 = \mathfrak{c}$ we can use the fact
that there are no cardinal numbers between $\mathfrak{c}$ and $\aleph_2$.
Also relying on the ``sandwich'' theorem, which says that if we can find a
superset with cardinality $n$ and a subset with the cardinality $n$,
then the cardinality of the set ``sandwiched'' in between must be $n$.
Consider $\mathfrak{c} \leq \mathfrak{c} \times \aleph_0 \leq \mathfrak{c}
    \times \mathfrak{c}$.  We can show that $\mathfrak{c} \times \aleph_0$ is at
least as big as $\mathfrak{c}$ since $\aleph_0$ is defined to be the
multiplicative identity.  We can also show that $\mathfrak{c} \times
    \mathfrak{c} = \mathfrak{c}$ by perusing Cantor's theorem which proves that
the cartesian product of an infinite set with itself has the same cardinality
as the initial set.  Thus $\mathfrak{c} \times \aleph_0 = \mathfrak{c}$.
\subsubsection{Answer 10}
\label{sec-1-4-3}
Observe that $\mathfrak{c}^{\mathfrak{c}}$ is the cardinality of the set of
all functions from a set of cardinality $\mathfrak{c}$ to itself, while
$2^{\mathfrak{c}}$ is the cardinality of the power-set of cardinality
$\mathfrak{c}$.  We know (from Cantor's diagonal argument) that $\abs{X} <
    \abs{P(X)}$ for all sets, this means that in particular $2^{\mathfrak{c}} <
    2^{2^\mathfrak{c}}$.  Since we know that cardinality of exponentiation is
non-decreasing in both arguments it follows that $\mathfrak{c}^{
    \mathfrak{c}} < 2^{2^{\mathfrak{c}}}$.  And because there are no cardinal
numbers between $X$ and $P(X)$ we can conclude that
$\mathfrak{c}^{\mathfrak{c}} = 2^{\mathfrak{c}}$.
% Emacs 25.0.50.1 (Org mode 8.2.2)
\end{document}