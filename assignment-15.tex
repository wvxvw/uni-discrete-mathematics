% Created 2015-06-05 Fri 12:54
\documentclass[11pt]{article}
\usepackage[utf8]{inputenc}
\usepackage[T1]{fontenc}
\usepackage{fixltx2e}
\usepackage{graphicx}
\usepackage{longtable}
\usepackage{float}
\usepackage{wrapfig}
\usepackage{rotating}
\usepackage[normalem]{ulem}
\usepackage{amsmath}
\usepackage{textcomp}
\usepackage{marvosym}
\usepackage{wasysym}
\usepackage{amssymb}
\usepackage{capt-of}
\usepackage{hyperref}
\tolerance=1000
\usepackage[utf8]{inputenc}
\usepackage[usenames,dvipsnames]{color}
\usepackage{commath}
\usepackage{tikz}
\usetikzlibrary{shapes,backgrounds}
\usepackage{marginnote}
\usepackage{listings}
\usepackage{color}
\usepackage{enumerate}
\hypersetup{urlcolor=blue}
\hypersetup{colorlinks,urlcolor=blue}
\setlength{\parskip}{16pt plus 2pt minus 2pt}
\renewcommand{\arraystretch}{1.6}
\author{Oleg Sivokon}
\date{\textit{<2015-06-01 Mon>}}
\title{Assignment 15, Discrete Mathematics}
\hypersetup{
 pdfauthor={Oleg Sivokon},
 pdftitle={Assignment 15, Discrete Mathematics},
 pdfkeywords={Discrete Mathematics, Assignment, Relations},
 pdfsubject={Second assignment in the course Discrete Mathematics},
 pdfcreator={Emacs 25.0.50.1 (Org mode 8.3beta)}, 
 pdflang={English}}
\begin{document}

\maketitle
\tableofcontents

\definecolor{codebg}{rgb}{0.96,0.99,0.8}
\definecolor{codestr}{rgb}{0.46,0.09,0.2}
\lstset{%
  backgroundcolor=\color{codebg},
  basicstyle=\ttfamily\scriptsize,
  breakatwhitespace=false,
  breaklines=false,
  captionpos=b,
  framexleftmargin=10pt,
  xleftmargin=10pt,
  framerule=0pt,
  frame=tb,
  keepspaces=true,
  keywordstyle=\color{blue},
  showspaces=false,
  showstringspaces=false,
  showtabs=false,
  stringstyle=\color{codestr},
  tabsize=2
}
\lstnewenvironment{maxima}{%
  \lstset{%
    backgroundcolor=\color{codebg},
    escapeinside={(*@}{@*)},
    aboveskip=20pt,
    captionpos=b,
    label=,
    caption=,
    showstringspaces=false,
    frame=single,
    framerule=0pt,
    basicstyle=\ttfamily\scriptsize,
    columns=fixed}}{}
}
\makeatletter
\newcommand{\verbatimfont}[1]{\renewcommand{\verbatim@font}{\ttfamily#1}}
\makeatother
\verbatimfont{\small}%
\clearpage

\section{Problems}
\label{sec:orgheadline11}

\subsection{Problem 1}
\label{sec:orgheadline3}
Given the alphabet \(\{0, 1, 2\}\) and the grammar rules which disallow
sequences \(\langle 00 \rangle\) and \(\langle 01 \rangle\) of the sequence
\(\{a_n\}\) which counts the possible products of the grammar:
\begin{enumerate}
\item Write \(a_0, a_1, a_2\), write recursive definition of the sequence.
\item Find a closed-form solution for \(\{a_n\}\).
\end{enumerate}

\subsubsection{Answer 1}
\label{sec:orgheadline1}
\begin{itemize}
\item \(a_0\) is the number of sequences of zero length, there's only one
such sequence.
\item \(a_1\) is the number of sequences of length one, there's exactly three
such sequences: \(\langle 0 \rangle\), \(\langle 1 \rangle\), \(\langle 2
      \rangle\).
\item \(a_2\) is the number of sequences of length two, here, for the first
time we encounter the restrictions imposed by the grammar.  Without
restriction, the number of sequences possible to form from an alphabet
of size 3 is \(3^2=9\), since two of these are not allowed, the total
number of sequences we can form is 7.
\end{itemize}

Now, let's construct the recursive definition.  Consider these two cases:
either the first element of the sequence is 2, or not.  If it is 2, then our
grammar doesn't restrict us, thus the number of possible grammar products is
to be \(a_{n-1}\).  Conversely, when the first number is either 0 or 1, the
further products can be only selected in a way that they don't contain zero,
i.e. there are only two possible ways to continue the sequence, which gives
us \(2 \cdot 2 a_{n-2}\).

In conclusion, the recursive definition is \(a_n = a_{n-1} + 4a_{n-2}\).
Verifying:
\begin{align*}
  a_0 &= 1 \\
  a_1 &= 3 \\
  a_2 &= a_1 + 4a_0 = 3 + 4 \cdot 1 = 7\;.
\end{align*}

\subsubsection{Answer 2}
\label{sec:orgheadline2}
In order to find the closed form solution, let's assume there exists
\(\lambda\) s.t. \(a_n = \lambda^n\).  Substituting our assumption back into
recursive formula gives:
\begin{align*}
  \lambda^n &= \lambda^{n-1} + 4\lambda^{n-2} &\iff \\
  \lambda^2 &= \lambda + 4                   &\iff \\
  \lambda^2 &- \lambda - 4 = 0               &\iff \\
  \lambda   &= \frac{-1 \pm \sqrt{1^2 - 4 \cdot (-4) \cdot 1}}{2 \cdot 1} \\
            &= \frac{-1 \pm \sqrt{17}}{2}\;.
\end{align*}

From here we see that there are two different terms contributing to the
function defining the sequence.  Lets denote the contribution these terms
make to the value as \(\alpha\) and \(\beta\).  Thus:
\begin{align*}
  a_n &= \alpha \frac{-1 + \sqrt{17}}{2} + \beta \frac{-1 - \sqrt{17}}{2} \\
      &\alpha + 4 \beta = 0 \\
      &\alpha \frac{-1 + \sqrt{17}}{2} + 4 \beta \frac{-1 - \sqrt{17}}{2} = 3\;. \\
\end{align*}

Solving these simultaneous equations gives: \(\alpha = -3, \beta = -\frac{3}{4}\).
Hence, the direct formula is:
\begin{align*}
  a_n &= 3 \left(\frac{-1 - \sqrt{17}}{2}\right)^n - 
  \frac{3}{4}\left(\frac{-1 + \sqrt{17}}{2}\right)^n\;.
\end{align*}

\subsection{Problem 2}
\label{sec:orgheadline5}
Find the number of solutions for \(x_1 + x_2 + x_3 + x_4 + x_5 = 24\)
provided two of the unknowns are odd natural numbers, the rest are even
natural numbers and neither unknown is equal to 0 or 1.

\subsubsection{Answer 3}
\label{sec:orgheadline4}
First, a simple way of counting the solutions.  We must deal at least a 3 to
two of the unknows and 2 to the other 3 of the unknowns.  This gives us \(2
    \cdot 3 + 3 \cdot 2\) of ``mandatory allocations''.  After these have been
allocated, we must distribute the remaining 12, but we can only distribute
it in pairs, so, all in all we end up with \({6 + 5 - 1 \choose 6} = 210\)
ways to allocate the ``points''.  Having done this, we count the number of
ways the solutions can be permuted: \(\frac{5!}{2!3!}=10\).  Hence, the final
answer, the number of solutions to the equation \(x_1 + x_2 + x_3 + x_4 + x_5
    = 24\) is \(10 \cdot 210 = 2100\).

The code to verify our calculations:
\lstset{language=prolog,label= ,caption= ,captionpos=b,numbers=none}
\begin{lstlisting}
:- use_module(library(clpfd)).

even(N) :- 0 is N mod 2.

assignment_5_2_helper(Unknowns) :-
    length(Unknowns, 5),
    Unknowns ins 2..24, sum(Unknowns, #=, 24),
    maplist(indomain, Unknowns),
    include(even, Unknowns, Evens),
    length(Evens, 3).

assignment_5_2 :-
    findall(X, assignment_5_2_helper(X), X),
    length(X, N),
    format('$$~p$$', [N]).
\end{lstlisting}

\(2100\)

To find what a generating function for this would look like we consider the
two groups of factors: the odd and the even ones.  Odd will contribute two
factors and even will contribute three, thus we have \(Odds^2 \cdot Evens^3\).
Now, odd factors can be written as \(x^3 + x^5 + x^7 + \hdots\) and, similarly
even factors are \(x^2 + x^4 + x^6 + \hdots\).  From the argument above we
know that there are 10 ways to arrange the unknowns, thus the final formula
for generating function is \(G(x) = 10(x^3 + x^5 + x^7 + \hdots)^2 (x^2 + x^4 +
    x^6 + \hdots)^3\).  The solution is given by the coefficient of \(x^{24}\).
Applying the following transformations:
\begin{align*}
  G(x) &= 10(x^3 + x^5 + x^7 + \hdots)^2 (x^2 + x^4 + x^6 + \hdots)^3 \\
       &= 10x^2(x^2 + x^4 + x^6 + \hdots)^2 (x^2 + x^4 + x^6 + \hdots)^3 \\
       &= 10x^2(x^2 + x^4 + x^6 + \hdots)^5 \\
       &= 10x^7(1 + x^2 + x^4 + \hdots)^5 \\
       &= \frac{10x^7}{(1 - x^2)^5} \\
       &= {-5 \choose 0} + {-5 \choose 1}10x^7 - {-5 \choose 2}10x^8 + \hdots
\end{align*}

We need the sixth term of these series, i.e. \({-5 \choose 6} = {5 + 6 - 1
   \choose 6} = 210\), multiplied by 10 (number of arrangements) gives the same
answer: 2100.

\subsection{Problem 3}
\label{sec:orgheadline8}
Joshua has to take pills:
\begin{itemize}
\item Against headache---at most 3 per day.
\item Energy pill---at most 3 per day.
\item Vitamin C---no restriction.
\item Vitamin B---no restriction.
\end{itemize}

Joshua is also required to take exactly \(n\) pills each day.  Let \(a_n\)
be the number of ways the pills can be combined.

\begin{enumerate}
\item Find a generating function for \(a_n\).
\item Find a closed-form formula for \(a_n\).
\end{enumerate}

\subsubsection{Answer 4}
\label{sec:orgheadline6}
We can create terms of the generating function in the following way: Since
Joshua has to take either no pills against headache, one, two or three, this
kind of pill will contribute the term \((1 + x + x^2 + x^3)\).  Similarly,
the energy pill.  Vitamins will contribute infinite sum terms:
\((1 + x + x^2 + x^3 + \hdots)\).  Cobining all together gives:
\(G(x) = (1 + x + x^2 + x^3)^2(1 + x + x^2 + x^3 + \hdots)^2\).

\subsubsection{Answer 5}
\label{sec:orgheadline7}
To find the closed-form formula for \(G(x)\) we can use the following identities:
\begin{align*}
  (1 + x + x^2 + x^3) &= \frac{1 - x^4}{1 - x} \\
  (1 + x + x^2 + x^3 + \hdots) &= \frac{1}{1 - x}
\end{align*}

Thus we can rewrite \(G\) as:
\begin{align*}
  G(x) &= \left(\frac{1 - x^4}{1 - x}\right)^2 \cdot \left(\frac{1}{1 - x}\right)^2 \\
       &= \frac{(1 - x^4)^2 \cdot (1 - x)^2}{(1 - x)^2} \\
       &= (1 - x^4)^2\;.
\end{align*}

\subsection{Problem 4}
\label{sec:orgheadline10}
Find the coefficient of \(x^{2m}\) in both sides of the following identity.
\begin{align*}
  \frac{(1 - x^2)^n}{(1 - x)^n} = (1 + x)^n\;.
\end{align*}

From this, derive the following idenity:
\begin{align*}
  \sum_{k=0}^? ?? = {n \choose 2m}\;.
\end{align*}

Verify the answer for the assignments: \(n = 5, m = 2\) and \(n = 5, m = 3\).

\subsubsection{Answer 6}
\label{sec:orgheadline9}
First, I will establish that the identity holds:
\begin{align*}
  \frac{(1 - x^2)^n}{(1 - x)^n} &= (1 - x^2)^n (1 + x + x^2 + x^3 + \hdots)^n \\
                               &= (1 - x + (1 - 1)x^3 + (1 - 1)x^4 + \hdots)^n \\
                               &= (1 + x)^n\;.
\end{align*}

Now, from binomial theorem:
\begin{align*}
  (1 + x)^n = \sum_{i=0}^\infty {n \choose i} x^i\;.
\end{align*}

Thus the coefficient of \(x^{2m} = {n \choose m}\).
\end{document}