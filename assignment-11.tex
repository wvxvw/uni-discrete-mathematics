% Created 2015-03-26 Thu 00:34
\documentclass[11pt]{article}
\usepackage[utf8]{inputenc}
\usepackage[T1]{fontenc}
\usepackage{fixltx2e}
\usepackage{graphicx}
\usepackage{longtable}
\usepackage{float}
\usepackage{wrapfig}
\usepackage{rotating}
\usepackage[normalem]{ulem}
\usepackage{amsmath}
\usepackage{textcomp}
\usepackage{marvosym}
\usepackage{wasysym}
\usepackage{amssymb}
\usepackage{hyperref}
\tolerance=1000
\usepackage[utf8]{inputenc}
\usepackage[usenames,dvipsnames]{color}
\usepackage[backend=bibtex, style=numeric]{biblatex}
\usepackage{commath}
\usepackage{tikz}
\usetikzlibrary{shapes,backgrounds}
\usepackage{marginnote}
\usepackage{listings}
\usepackage{color}
\usepackage{enumerate}
\hypersetup{urlcolor=blue}
\hypersetup{colorlinks,urlcolor=blue}
\addbibresource{bibliography.bib}
\setlength{\parskip}{16pt plus 2pt minus 2pt}
\definecolor{codebg}{rgb}{0.96,0.99,0.8}
\definecolor{codestr}{rgb}{0.46,0.09,0.2}
\author{Oleg Sivokon}
\date{\textit{<2015-03-21 Sat>}}
\title{Assignment 11, Discrete Mathematics}
\hypersetup{
  pdfkeywords={Discrete Mathematics, Assignment, Set Theory},
  pdfsubject={First asssignment in the course Discrete Mathematics},
  pdfcreator={Emacs 25.0.50.1 (Org mode 8.2.2)}}
\begin{document}

\maketitle
\tableofcontents


\lstset{ %
  backgroundcolor=\color{codebg},
  basicstyle=\ttfamily\scriptsize,
  breakatwhitespace=false,         % sets if automatic breaks should only happen at whitespace
  breaklines=false,
  captionpos=b,                    % sets the caption-position to bottom
  commentstyle=\color{mygreen},    % comment style
  framexleftmargin=10pt,
  xleftmargin=10pt,
  framerule=0pt,
  frame=tb,                        % adds a frame around the code
  keepspaces=true,                 % keeps spaces in text, useful for keeping indentation of code (possibly needs columns=flexible)
  keywordstyle=\color{blue},       % keyword style
  showspaces=false,                % show spaces everywhere adding particular underscores; it overrides 'showstringspaces'
  showstringspaces=false,          % underline spaces within strings only
  showtabs=false,                  % show tabs within strings adding particular underscores
  stringstyle=\color{codestr},     % string literal style
  tabsize=2,                       % sets default tabsize to 2 spaces
}

\clearpage

\section{Problems}
\label{sec-1}

\subsection{Problem 1}
\label{sec-1-1}
In all questions that follow, establish whether $x \in y$ or $x \subseteq y$.
\begin{enumerate}
\item $x = \{1, 2\}, y = \{1, 2, 3\}$.
\item $x = \{3\}, y = \{\{1\}, \{2\}, \{3\}\}$.
\item $x = \{1, 2\}, y = \{\{1, 2\}, 3\}$.
\item $x = \{1, 3\}, y = \{\{1, 2\}, 3\}$.
\item $x = \emptyset, y = \emptyset$.
\item $x = \{\emptyset\}, y = \{\emptyset\}$.
\item $x = \{1\}, y = \{1, 2\}$.
\item $x = \emptyset, y = P(\{1, 2, 3\})$.
\end{enumerate}

\subsubsection{Answer 1}
\label{sec-1-1-1}
\begin{enumerate}
\item $x \subseteq y$.
\item $x \in y$.
\item $x \in y$.
\item $x \not \in y \land x \not \subseteq y$.
\item $x \subseteq y$.
\item $x \subseteq y$.
\item $x \subseteq y$.
\item $x \subseteq y$.
\end{enumerate}
\subsection{Problem 2}
\label{sec-1-2}
Prove or disprove:
\begin{enumerate}
\item $(A \setminus B) \setminus B = A \setminus B$.
\item $A \setminus (B \setminus A) = A$.
\item $A \subseteq P(A)$.
\end{enumerate}

\subsubsection{Answer 2}
\label{sec-1-2-1}
The statement is true.  Suppose for contradiction that there is a $b \in A
    \setminus B$ s.t. $b \not \in (A \setminus B) \setminus B$.  Then, we know
that $b \in A$ and $b \not \in B$.  But we assumed $b \not \in A \lor b \in
    B$.  This is a contradiction, hence $(A \setminus B) \setminus B = A
    \setminus B$.  Additionally, it is easy to see that set subtraction is
idempotent (i.e. if applied repeatedly, will yield the same result after
first application).
\subsubsection{Answer 3}
\label{sec-1-2-2}
The statement is true.  Suppose for contradiction there was an $a \in A$,
which is not in $A \setminus (B \setminus A)$.  Then such $a$ would have
to be in $A$, but not in $(B \setminus A)$.  I will now show that no elements
of $(B \setminus A)$ is in $A$, thus subtracting this group would not
remove any elements of $A$, i.e. $A \cap (B \setminus A) = \emptyset$.
Suppose $b$ was in $A$ and in $(B \setminus A)$ at the same time, then
it would have to be in $A$, but $A$ is excluded from $(B \setminus A)$.
This is in contradiction to the previous assumtion.  Therefore
$A \setminus (B \setminus A) = A$.
\subsubsection{Answer 4}
\label{sec-1-2-3}
This statement isn't true in general.  An example which disproves it would
be an assignment: $A=\{1, 2\}$.  $P(A)=\{\emptyset, \{1\}, \{2\}, \{1, 2\}\}$.
$A \not \subseteq P(A)$.
\subsection{Problem 3}
\label{sec-1-3}
Prove the following identity using $A \setminus B = A \cap B'$:
$(A_1 \cup A_2) \setminus (B_1 \cap B_2) = (A_1 \cup B_1) \cup (A_1 \cup B_2)
   \cup (A_2 \cup B_1) \cup (A_2 \cup B_2)$.

\subsubsection{Answer 5}
\label{sec-1-3-1}
The proof doesn't require discussion.  It is accomplished by simple
set-theoretical manipulations.
\begin{equation*}
  \begin{aligned}
    (A_1 \cup A_2) \setminus (B_1 \cap B_2) &\iff
    &\textrm{Given} \\
    A_1 \setminus (B_1 \cap B_2) \cup A_2 \setminus (B_1 \cap B_2) &\iff
    &\textrm{Union over subtraction} \\
    A_1 \cap (B_1 \cap B_2)' \cup A_2 \cap (B_1 \cap B_2)' &\iff
    &\textrm{Required identity} \\
    A_1 \cap (B_1' \cup B_2') \cup A_2 \cap (B_1' \cup B_2') &\iff
    &\textrm{De Morgan's law} \\
    (A_1 \cap B_1') \cup (A_1 \cap B_2') \cup
    (A_2 \cap B_1') \cup (A_2 \cap B_2') &\iff
    &\textrm{Union over subtraction} \\
    (A_1 \setminus B_1) \cup (A_1 \setminus B_2)
    \cup (A_2 \setminus B_1) \cup (A_2 \setminus B_2) &\iff
    &\textrm{Required identity}
  \end{aligned}
\end{equation*}
\subsection{Problem 4}
\label{sec-1-4}
Given $\forall n \in \mathbb{N}: A_n = \{x \in \mathbb{R} | 4 \leq x \leq
   2n + 2 \}$ and $\forall n \in \mathbb{N}: B_n = A_{n+1} \setminus A_n$.
\emph{For the purpose of this exercise} $\mathbb{N}$ \emph{contains zero.}
\begin{enumerate}
\item Calculate $A_0, A_1, A_2, A_3$ and $B_0, B_1, B_2$.
\item Write closed-form formula for $B_n$.
\item Calculate $\bigcup_{2 \leq n \in \mathbb{N}}B_n$.  Prove the result
using set containment relation.
\item Using general rules for set subtraction and union as well as De Morgan
laws as applied to universal and existential quantifiers, prove the
generalized formulae for $\bigcup_{i \in I}(A')$ and 
$\bigcap_{i \in I}(A')$.
\item Let $D_n = \mathbb{R} \setminus B_n$.  Calculate 
      $\bigcup_{2 \leq n \in \mathbb{N}}D_n$.
\end{enumerate}

\subsubsection{Answer 6}
\label{sec-1-4-1}
It is easy to do the calculation by hand, but to make it more interesting
I wrote some Prolog code to do it.  An intuitive way to see what is going
on is to observe that every consequent $A$ will contain the entire previous
set and two more members, which are greater than the maximal element of
the previously collected set.

\lstset{language=prolog,numbers=none}
\begin{lstlisting}
:- use_module(library(clpfd)).

set_A(N, Set) :- 
    High is N * 2 + 2,
    X in 4..High,
    findall(X, indomain(X), Set).

set_B(N, Set) :-
    set_A(N, A_n),
    N1 is N + 1,
    set_A(N1, A_n1),
    subtract(A_n1, A_n, Set).

all_sets(N, Pred, Answer) :-
    X in 1..N, indomain(X),
    call(Pred, X, Answer).

zip(X, Y, [X, Y]).

join(_, [X], X) :- !.
join(Sep, [X | Xs], S) :-
    join(Sep, Xs, Sx),
    string_concat(Sep, Sx, Sy),
    string_concat(X, Sy, S).

question_1(Na, Nb, As, Bs):-
    findall(As, all_sets(Na, set_A, As), As),
    findall(Bs, all_sets(Nb, set_B, Bs), Bs),
    X in 1..Na,
    Y in 1..Nb,
    findall(X, indomain(X), Nas),
    findall(Y, indomain(Y), Nbs),
    maplist(join(','), As, Jas),
    maplist(join(','), Bs, Jbs),
    maplist(zip, Nas, Jas, Zas),
    maplist(zip, Nbs, Jbs, Zbs),
    maplist(format('$A_~p=\\{~p\\}$~n~n'), Zas),
    maplist(format('$B_~p=\\{~p\\}$~n~n'), Zbs).
\end{lstlisting}

$A_1=\{4\}$

$A_2=\{4,5,6\}$

$A_3=\{4,5,6,7,8\}$

$B_1=\{5,6\}$

$B_2=\{7,8\}$

$B_3=\{9,10\}$

\subsubsection{Answer 7}
\label{sec-1-4-2}
$B = \{x \in \mathbb{R} | 2(n+1) < x \leq 2(n+2)\}$.
\subsubsection{Answer 8}
\label{sec-1-4-3}
First, let me make the claim that $\bigcup_{2 \leq n \in \mathbb{N}}=A_{n+1}$.
Below, is the proof that doesn't require division in two cases (it peruses
the definition of $B_n$ and the general technique of extraction of the last
term of a sequence).

\begin{equation*}
  \begin{aligned}
    \bigcup_{2 \leq n \in \mathbb{N}} B_n &=
    \bigcup_{2 \leq n \in \mathbb{N}} A_{n+1} \setminus A_n \\
    &= \bigcup_{2 \leq n \in \mathbb{N}} A_{n+1} \setminus
    \bigcup_{2 \leq n \in \mathbb{N}} A_n \\
    &= \bigcup_{2 \leq n \in \mathbb{N}} A_n \setminus
    \bigcup_{2 \leq n \in \mathbb{N}} A_n \cup A_{n+1} \\
    &= A_{n+1}
  \end{aligned}
\end{equation*}
\subsubsection{Answer 9}
\label{sec-1-4-4}
First, let me reiterate De Morgan's law for first-order quantifiers:
\begin{equation*}
  \begin{aligned}
    \lnot \forall x. \phi \iff \exists x. \lnot \phi.
  \end{aligned}
\end{equation*}

Combining it with the definitions of union of complement sets and
intersection of complement sets gives us the following proof:

\begin{equation*}
  \begin{aligned}
    x \in \bigcap_{i \in I} (A_i)'
    &\iff \forall i (i \in I \implies x \in A') \\
    &\iff \exists i (i \in I \implies x \in A)' \\
    &\iff x \in (\bigcup_{i \in I} A_i)'.
  \end{aligned}
\end{equation*}

\emph{Note that the proof treats negation equivalently to complementation.}

The proof for the union is symmetrical:
\begin{equation*}
  \begin{aligned}
    x \in \bigcup_{i \in I} (A_i)'
    &\iff \exists i (i \in I \implies x \in A') \\
    &\iff \forall i (i \in I \implies x \in A)' \\
    &\iff x \in (\bigcap_{i \in I} A_i)'.
  \end{aligned}
\end{equation*}
\subsubsection{Answer 10}
\label{sec-1-4-5}
Reusing definitions and conclusions derived above gives:
\begin{equation*}
  D_n = \mathbb{R} \setminus B_n = \mathbb{R} \cap (B_n)'.
\end{equation*}
\begin{equation*}
  \begin{aligned}
    \bigcap_{2 \leq n \in \mathbb{N}} D_n
    &= \bigcap_{2 \leq n \in \mathbb{N}} \mathbb{R} \cap (B_n)' \\
    &= \mathbb{R} \cap \bigcap_{2 \leq n \in \mathbb{N}} (B_n)' \\
    &= \mathbb{R} \cap (\bigcup_{2 \leq n \in \mathbb{N}} B_n)' \\
    &= \mathbb{R} \cap (\bigcup_{2 \leq n \in \mathbb{N}} A_{n+1} \setminus A_n)' \\
    &= \mathbb{R} \cap (\bigcup_{2 \leq n \in \mathbb{N}} A_{n+1}
    \setminus \bigcup_{2 \leq n \in \mathbb{N}} A_n)' \\
    &= \mathbb{R} \cap (\bigcup_{2 \leq n \in \mathbb{N}} A_{n+1}
    \cap (\bigcup_{2 \leq n \in \mathbb{N}} A_n)')' \\
    &= \mathbb{R} \cap (\bigcup_{2 \leq n \in \mathbb{N}} A_{n+1})'
    \cup (\bigcup_{2 \leq n \in \mathbb{N}} A_n)' \\
    &= \mathbb{R} \cap (A_{n+1})' \\
    &= \mathbb{R} \setminus A_{n+1}
  \end{aligned}
\end{equation*}
% Emacs 25.0.50.1 (Org mode 8.2.2)
\end{document}