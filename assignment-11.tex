% Created 2015-03-21 Sat 17:09
\documentclass[11pt]{article}
\usepackage[utf8]{inputenc}
\usepackage[T1]{fontenc}
\usepackage{fixltx2e}
\usepackage{graphicx}
\usepackage{longtable}
\usepackage{float}
\usepackage{wrapfig}
\usepackage{rotating}
\usepackage[normalem]{ulem}
\usepackage{amsmath}
\usepackage{textcomp}
\usepackage{marvosym}
\usepackage{wasysym}
\usepackage{amssymb}
\usepackage{hyperref}
\tolerance=1000
\usepackage[utf8]{inputenc}
\usepackage[usenames,dvipsnames]{color}
\usepackage[backend=bibtex, style=numeric]{biblatex}
\usepackage{commath}
\usepackage{tikz}
\usetikzlibrary{shapes,backgrounds}
\usepackage{marginnote}
\usepackage{listings}
\usepackage{color}
\usepackage{enumerate}
\hypersetup{urlcolor=blue}
\hypersetup{colorlinks,urlcolor=blue}
\addbibresource{bibliography.bib}
\setlength{\parskip}{16pt plus 2pt minus 2pt}
\definecolor{codebg}{rgb}{0.96,0.99,0.8}
\definecolor{codestr}{rgb}{0.46,0.09,0.2}
\author{Oleg Sivokon}
\date{\textit{<2015-03-21 Sat>}}
\title{Assignment 11, Discrete Mathematics}
\hypersetup{
  pdfkeywords={Discrete Mathematics, Assignment, Set Theory},
  pdfsubject={First asssignment in the course Discrete Mathematics},
  pdfcreator={Emacs 25.0.50.1 (Org mode 8.2.2)}}
\begin{document}

\maketitle
\tableofcontents


\lstset{ %
  backgroundcolor=\color{codebg},
  basicstyle=\ttfamily\scriptsize,
  breakatwhitespace=false,         % sets if automatic breaks should only happen at whitespace
  breaklines=false,
  captionpos=b,                    % sets the caption-position to bottom
  commentstyle=\color{mygreen},    % comment style
  framexleftmargin=10pt,
  xleftmargin=10pt,
  framerule=0pt,
  frame=tb,                        % adds a frame around the code
  keepspaces=true,                 % keeps spaces in text, useful for keeping indentation of code (possibly needs columns=flexible)
  keywordstyle=\color{blue},       % keyword style
  showspaces=false,                % show spaces everywhere adding particular underscores; it overrides 'showstringspaces'
  showstringspaces=false,          % underline spaces within strings only
  showtabs=false,                  % show tabs within strings adding particular underscores
  stringstyle=\color{codestr},     % string literal style
  tabsize=2,                       % sets default tabsize to 2 spaces
}

\clearpage

\section{Problems}
\label{sec-1}

\subsection{Problem 1}
\label{sec-1-1}
In all questions that follow, establish whether $x \in y$ or $x \subseteq y$.
\begin{enumerate}
\item $x = \{1, 2\}, y = \{1, 2, 3\}$.
\item $x = \{3\}, y = \{\{1\}, \{2\}, \{3\}\}$.
\item $x = \{1, 2\}, y = \{\{1, 2\}, 3\}$.
\item $x = \{1, 3\}, y = \{\{1, 2\}, 3\}$.
\item $x = \emptyset, y = \emptyset$.
\item $x = \{\emptyset\}, y = \{\emptyset\}$.
\item $x = \{1\}, y = \{1, 2\}$.
\item $x = \emptyset, y = P(\{1, 2, 3\})$.
\end{enumerate}

\subsubsection{Answer 1}
\label{sec-1-1-1}
\begin{enumerate}
\item $x \subseteq y$.
\item $x \in y$.
\item $x \in y$.
\item $x \not \in y \land x \not \subseteq y$.
\item $x \subseteq y$.
\item $x \subseteq y$.
\item $x \subseteq y$.
\item $x \subseteq y$.
\end{enumerate}
\subsection{Problem 2}
\label{sec-1-2}
Prove or disprove:
\begin{enumerate}
\item $(A \setminus B) \setminus B = A \setminus B$.
\item $A \setminus (B \setminus A) = A$.
\item $A \subseteq P(A)$.
\end{enumerate}

\subsubsection{Answer 2}
\label{sec-1-2-1}
The statement is true.  Suppose for contradiction that there is a $b \in A
    \setminus B$ s.t. $b \not \in (A \setminus B) \setminus B$.  Then, we know
that $b \in A$ and $b \not \in B$.  But we assumed $b \not \in A \lor b \in
    B$.  This is a contradiction, hence $(A \setminus B) \setminus B = A
    \setminus B$.  Additionally, it is easy to see that set subtraction is
idempotent (i.e. if applied repeatedly, will yield the same result after
first application).
\subsubsection{Answer 3}
\label{sec-1-2-2}
The statement is true.  Suppose for contradiction there was an $a \in A$,
which is not in $A \setminus (B \setminus A)$.  Then such $a$ would have
to be in $A$, but not in $(B \setminus A)$.  I will now show that no elements
of $(B \setminus A)$ is in $A$, thus subtracting this group would not
remove any elements of $A$, i.e. $A \cap (B \setminus A) = \emptyset$.
Suppose $b$ was in $A$ and in $(B \setminus A)$ at the same time, then
it would have to be in $A$, but $A$ is excluded from $(B \setminus A)$.
This is in contradiction to the previous assumtion.  Therefore
$A \setminus (B \setminus A) = A$.
\subsubsection{Answer 4}
\label{sec-1-2-3}
This statement isn't true in general.  An example which disproves it would
be an assignment: $A=\{1, 2\}$.  $P(A)=\{\emptyset, \{1\}, \{2\}, \{1, 2\}\}$.
$A \not \subseteq P(A)$.
% Emacs 25.0.50.1 (Org mode 8.2.2)
\end{document}