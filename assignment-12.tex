% Created 2015-04-19 Sun 00:52
\documentclass[11pt]{article}
\usepackage[utf8]{inputenc}
\usepackage[T1]{fontenc}
\usepackage{fixltx2e}
\usepackage{graphicx}
\usepackage{longtable}
\usepackage{float}
\usepackage{wrapfig}
\usepackage{rotating}
\usepackage[normalem]{ulem}
\usepackage{amsmath}
\usepackage{textcomp}
\usepackage{marvosym}
\usepackage{wasysym}
\usepackage{amssymb}
\usepackage{hyperref}
\tolerance=1000
\usepackage[utf8]{inputenc}
\usepackage[usenames,dvipsnames]{color}
\usepackage[backend=bibtex, style=numeric]{biblatex}
\usepackage{commath}
\usepackage{tikz}
\usetikzlibrary{shapes,backgrounds}
\usepackage{marginnote}
\usepackage{listings}
\usepackage{color}
\usepackage{enumerate}
\hypersetup{urlcolor=blue}
\hypersetup{colorlinks,urlcolor=blue}
\addbibresource{bibliography.bib}
\setlength{\parskip}{16pt plus 2pt minus 2pt}
\definecolor{codebg}{rgb}{0.96,0.99,0.8}
\definecolor{codestr}{rgb}{0.46,0.09,0.2}
\author{Oleg Sivokon}
\date{\textit{<2015-04-05 Sun>}}
\title{Assignment 12, Discrete Mathematics}
\hypersetup{
  pdfkeywords={Discrete Mathematics, Assignment, Relations},
  pdfsubject={Second asssignment in the course Discrete Mathematics},
  pdfcreator={Emacs 25.0.50.1 (Org mode 8.2.2)}}
\begin{document}

\maketitle
\tableofcontents


\lstset{ %
  backgroundcolor=\color{codebg},
  basicstyle=\ttfamily\scriptsize,
  breakatwhitespace=false,         % sets if automatic breaks should only happen at whitespace
  breaklines=false,
  captionpos=b,                    % sets the caption-position to bottom
  commentstyle=\color{mygreen},    % comment style
  framexleftmargin=10pt,
  xleftmargin=10pt,
  framerule=0pt,
  frame=tb,                        % adds a frame around the code
  keepspaces=true,                 % keeps spaces in text, useful for keeping indentation of code (possibly needs columns=flexible)
  keywordstyle=\color{blue},       % keyword style
  showspaces=false,                % show spaces everywhere adding particular underscores; it overrides 'showstringspaces'
  showstringspaces=false,          % underline spaces within strings only
  showtabs=false,                  % show tabs within strings adding particular underscores
  stringstyle=\color{codestr},     % string literal style
  tabsize=2,                       % sets default tabsize to 2 spaces
}

\clearpage

\section{Problems}
\label{sec-1}

\subsection{Problem 1}
\label{sec-1-1}
Let $M$ be the set of all relations over $A=\{1, 2, 3\}$.
\begin{enumerate}
\item How many members does $M$ have?
\item Let $S$ be a set of relations over $M$, defined as follows:
\begin{equation*}
  S=\{R_1R_2 | R_1, R_2 \in M \land R_1R_2 = R_2R_1\}.
\end{equation*}

Show that $S$ is not an equivalence relation.
\end{enumerate}

\subsubsection{Answer 1}
\label{sec-1-1-1}
The number of elements of $M$ is $2^{\left|A\right|^2} = 2^{3^2} = 2^9
    = 512$.  This can be proved from a partial sum of a recurence: $a_{n+1} =
    a_n + 2n - 1$, which describes the maximum number of ordered pairs possible
to create from $n$ elements and $b_n = 2b_{n-1}$ recurrence which describes
the number of elements of a powerset.  Since one can see that the number
of relations over a set is exactly the number of ways to subset the set of
all ordered pairs possible to create from that set, the final answer is
the composition of both recurrences: $c_n=b_n \circ a_n$.

Below I've implemented the counting algorithm in Prolog:

\lstset{language=prolog,numbers=none}
\begin{lstlisting}
powerset(Set, Result) :-
    powerset_helper(Set, [[]], Result).
powerset_helper([], X, X) :- !.
powerset_helper([X | Xs], In, Result) :-
    findall(Z, (member(Y, In), append([X], Y, Z)), Z),
    append(In, Z, Next),
    powerset_helper(Xs, Next, Result).

cross(Set, (A, B)) :- member(A, Set), member(B, Set).

pairs(Set, Pairs) :- findall(X, cross(Set, X), Pairs).

question_1(A) :-
    pairs(A, Pairs),
    powerset(Pairs, M),
    length(M, Len),
    format("$\\left|M\\right| = ~p$~n", [Len]).
\end{lstlisting}

$\left|M\right| = 512$
\subsubsection{Answer 2}
\label{sec-1-1-2}
I will prove that $S$ is not an equivalence using the definition of
equivalence which states that a relation is an equivalence if it is
\textbf{relfexive}, \textbf{symmetric} and \textbf{transitive}.  It is easy to see that the
definition of reflexivity requires that \emph{all} members of $M$ be present in
the relation, but, for example, $\{(1, 2)\}$ is absent from $S$.  Suppose,
for contradiction, it was present in $S$, then it would imply that there
exists a pair $\{(a, b)\} \in M$ s.t. $(\{(1, 2)\}, \{(a, b)\}) \in S$ and
$(\{(a, b)\}, \{(1, 2)\}) \in\nobreak S$.  By looking at the first and the
last members of the two members of $S$ we know that $\{(1, 2)\} \circ \{(a,
    b)\} = \{(1, b)\}$ and $\{(a, b)\} \circ \{(1, 2)\} = \{(a, 2)\}$, in other
words it is necessary that: $\{(1, 2)\} \circ \{(a, b)\} = \{(1, b)\} =
    \{(1, 2)\} = \{(a, 2)\} = \{(a, b)\} \circ \{(1, 2)\}$, which in turn, means
that $b=2$ and $a=1$.  Plugging these values back into original equilaty
gives $\{(1, 2)\} \circ \{(1, 2)\} = \{(1, 2)\}$, which is obviously false.
Hence, by contradiction, the proof is complete.
\subsection{Problem 2}
\label{sec-1-2}
Given a set $A=\{1, 2, 3\}$ and $M$ a set of all reations over $A$.  Let also
$s : M \to M$ be a function assigning to each $R \in M$ its transitive
closure ($R^{+}$).  Prove or disprove:
\begin{enumerate}
\item $s$ is one-to-one.
\item $s$ is onto.
\item For all $R_1,R_2 \in M$ it holds that $s(R_1R_2) = s(R_1)s(R_2)$.
\item For all $R \in M$ it holds that $s(R) = s(s(R))$.
\end{enumerate}

\subsubsection{Answer 3}
\label{sec-1-2-1}
A function is one-to-one when if $s(x)=s(y)$ then it must also be that
$x=y$.  From definition of transitive closure it follows that it may be
closed over more than one set (i.e. it must be its own subset but may have
more subsets which are not transitive relations).  Thus it's not true that
$s(x) = s(y) \implies x=y$. To be more concrete, suppose this example:
$R_1=\{(1, 2), (1, 3)\}$ and $R_2=\{(1, 2), (2, 3)\}$.
Then $s(R_1)=\{(1, 2), (1, 3), (2, 3)\}$ and $s(R_2)=\{(1, 2), (2, 3), (1, 3)\}$.
Thus $s$ is not one-to-one.
\subsubsection{Answer 4}
\label{sec-1-2-2}
Recall the definition of a function which is onto: for each element in the
function's co-domain there exists an element in its domain.  Using the
definition of transitive closure, every such closure contains at least one
relation (itself), thus $s$ has an element in its domain for each element in
its co-domain.  Thus the claim is true.
\subsubsection{Answer 5}
\label{sec-1-2-3}
This is not in general true, consider $R_1$ to be $\{(1,1), (1,2), (1,3)\}$
and $R_2=\{(3,3)\}$.  Then:
\begin{eqnarray*}
  s(R_1R_2)    &= s(\{(1,3)\})                                   &= \{(1,3)\} \\
  s(R_1)s(R_2) &= \{(1,1), (1,2), (2,3), (1,3)\} \circ \{(3,3)\} &= \{(2,3), (1,3)\}.
\end{eqnarray*}
\subsubsection{Answer 6}
\label{sec-1-2-4}
It is true that $s$ is \emph{idempotent}. This is so because for any transitive
relation its transitive closure is equal to it and the co-domain of $s$ is
defined to contain only transitive relations.
\subsection{Problem 3}
\label{sec-1-3}
Let $F$ be the set of functions from $\mathbb{N}$ to itself.  Define relation
$K$ over $F$ s.t. $f, g \in F, (f, g) \in K \iff \forall n \in \mathbb{N} f(n) \leq g(n)$.
\begin{enumerate}
\item Prove that $K$ is a partially ordered.
\item Prove that $K$ isn't totally ordered.
\item Are there maximal members in $K$? Is there a largest member in $K$?
\item Are there minimal members in $K$? Is there a smallest member in $K$?
\item Prove that for any $f \in F$ exists $g \in F$ bounding it from above.
Prove there are more than one such $g$.
\end{enumerate}

\subsubsection{Answer 7}
\label{sec-1-3-1}
Partially ordered sets are defined to be reflexive, antisymmetrical and
transitive. Let's verify all these properties:
\begin{description}
\item[{Reflexitivty}] $f(n) \leq f(n)$ is true because the co-domains of $f$ and
$g$ are the set of natural numbers, for which this property also holds, besides,
we required that the inequality holds for each value of the functions taken
pair-wise.
\item[{Antisymmetry}] whenever $f(n) \leq g(n)$ and $f(n) \leq g(n)$
then $f(n) = g(n)$.  By the same reasoning as above, $K$ is antisymmetrical.
\item[{Transitivity}] if $f(n) \leq g(n)$ and $q(n) \leq f(n)$, then $q(n) \leq g(n)$.
And, again, since the co-domain of all these functions is the natural numbers
transitivity holds.
\end{description}

Thus $K$ is partially ordered.
\subsubsection{Answer 8}
\label{sec-1-3-2}
To say that $K$ isn't totally ordered is to say that there exists a pair of
functions $f'$ and $g'$ such that for them neither $f'(n) \leq g'(n)$ nor
$g'(n) \leq f'(n)$ for all $n \in \mathbb{N}$.  Any two functions, whose graphs
cross will do the job (before the interesection point one of the functions
will be greater than the other and after the intersection the relation will
change sides).  So, for example functions $g'(n)=n^2$ and $f'(n)=n+2$ are
$g'(n) < f'(n)$ when $n < 2$ and $g'(n) > f'(n)$ otherwise.
\subsubsection{Answer 9}
\label{sec-1-3-3}
No, there isn't a maximal element in $K$ since there isn't a maximal element
in the natrual numbers (and natural numbers are the co-domain of the functions
used to construct $K$). Consequently, there isn't a largest element in $K$.
\subsubsection{Answer 10}
\label{sec-1-3-4}
However, there is a minimal element in $K$ (and subsequently the smallest one).
It is $(x(n)=0, x(n)=0)$.  It is easy to see that this element is in $K$,
since it is a function from $\mathbb{N}$ to itself, also both elements of the
pair adhere to the condition that they'd be no greater for all $n \in \mathbb{N}$.
It is also easy to see that there is no pair which is not strictly greater
than this element.  Suppose, for contradiction, there was such a pair, then
for some natural number $n$, the $x'(n) \leq x(n) = 0$, but 0 is the smallest
element of $\mathbb{N}$, contrary to assumed.  Thus there is a smallest element
and the set of minimal elements (consisting of the smallest element).
\subsubsection{Answer 11}
\label{sec-1-3-5}
Suppose for contradiction there was an $x(n)$ which wasn't bounded above by any
$y(n)$.  Then its value at $n$ would be some $m \in \mathbb{N}$, but we can
construct $y(n)=x(n)+1$, in contradiction to our initial assumption.  Thus
every element in $K$ must be bounded from above.  Observe also that the bounds
themselves are members of $K$, and since they are, then each bound has its own
bound, and by transitivity of the relation ``the upper bound of'' we can see
that every member of $K$ has infinitely many such bounds.
\subsection{Problem 4}
\label{sec-1-4}
\begin{enumerate}
\item Prove by induction that the following function definitions are equivalent:
      $f(0)=0$, $f(1)=10$, $f(n+2)=f(n+1)+6f(n)$. And $f(n)=2*3^n+(-2)^{n+1}$.
\item Is the function defined above onto?
\end{enumerate}

\subsubsection{Answer 12}
\label{sec-1-4-1}
Using mathematical induction, let's first verify the base step:
$f(0)=0$ and $f(0)=2*3^0+(-2)^{1}=2-2=0$.

Inductive step will first establish the relation between three subsequent
terms of the sequence, and then verify that the same relation holds for
both definition of $f$.
\begin{equation*}
  \begin{split}
    f(n + 3) &= f(n + 2) + 6f(n + 1) \\
             &= f(n + 1) + 6f(n) + 6f(n + 1) \\
             &= 7f(n + 1) + 6f(n) \\
    f(n + 3) - f(n + 2) &= 7f(n + 1) + 6f(n) - f(n + 1) - 6f(n) \\
                        &= 6f(n)
  \end{split}
\end{equation*}

Now, I will use the induction hypothesis to demonstrate that $f(n+2)-f(n+1)=f(n)$:
\begin{equation*}
  \begin{split}
    & f(n + 2) - f(n + 1) - 6f(n) = 0 \\
    & 2*3^{n+2} + (-2)^{n+3} - 2*3^{n+1} - (-2)^{n+2} - 6*2*3^n - (-2)^{n+1} = 0 \\
    & 2*3^{n+2} - 2*3^{n+1} - 6*2*3^n + (-2)^{n+3} - (-2)^{n+2} - (-2)^{n+1} = 0 \\
    & 2*3^{n+2} - 6*3^n - 6*2*3^n + \\
    & \hspace{4em} + (-2)*(-2)*(-2)^{n+1} - (-2)(-2)^{n+1} - 6*(-2)^{n+1} = 0 \\
    & 2*3^{n+2} - 3^n(6 + 6*2) + (-2)^{n+1}(4 - (-2) - 6) = 0 \\
    & 2*3^{n+2} - 3^n*2*9 + (-2)^{n+1}*0 = 0 \\
    & 0 + 0 = 0.
  \end{split}
\end{equation*}

Having showed that the induction step holds too, using the principle of mathematical
induction the proof is complete.
\subsubsection{Answer 13}
\label{sec-1-4-2}
Recall that in order to demonstrate that the function is not onto it is enough to find
an element in its co-domain, which is not in its domain.  Suppose this function had
a value in range $(0, 10)$, but since multiplication is preserving inequality and
is non-decreasing, the value would have to be between zero and one, but there are no
natural numbers between zero and one, hence the function is not onto.
% Emacs 25.0.50.1 (Org mode 8.2.2)
\end{document}