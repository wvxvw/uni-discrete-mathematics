% Created 2015-04-06 Mon 22:26
\documentclass[11pt]{article}
\usepackage[utf8]{inputenc}
\usepackage[T1]{fontenc}
\usepackage{fixltx2e}
\usepackage{graphicx}
\usepackage{longtable}
\usepackage{float}
\usepackage{wrapfig}
\usepackage{rotating}
\usepackage[normalem]{ulem}
\usepackage{amsmath}
\usepackage{textcomp}
\usepackage{marvosym}
\usepackage{wasysym}
\usepackage{amssymb}
\usepackage{hyperref}
\tolerance=1000
\usepackage[utf8]{inputenc}
\usepackage[usenames,dvipsnames]{color}
\usepackage[backend=bibtex, style=numeric]{biblatex}
\usepackage{commath}
\usepackage{tikz}
\usetikzlibrary{shapes,backgrounds}
\usepackage{marginnote}
\usepackage{listings}
\usepackage{color}
\usepackage{enumerate}
\hypersetup{urlcolor=blue}
\hypersetup{colorlinks,urlcolor=blue}
\addbibresource{bibliography.bib}
\setlength{\parskip}{16pt plus 2pt minus 2pt}
\definecolor{codebg}{rgb}{0.96,0.99,0.8}
\definecolor{codestr}{rgb}{0.46,0.09,0.2}
\author{Oleg Sivokon}
\date{\textit{<2015-04-05 Sun>}}
\title{Assignment 12, Discrete Mathematics}
\hypersetup{
  pdfkeywords={Discrete Mathematics, Assignment, Relations},
  pdfsubject={Second asssignment in the course Discrete Mathematics},
  pdfcreator={Emacs 25.0.50.1 (Org mode 8.2.2)}}
\begin{document}

\maketitle
\tableofcontents


\lstset{ %
  backgroundcolor=\color{codebg},
  basicstyle=\ttfamily\scriptsize,
  breakatwhitespace=false,         % sets if automatic breaks should only happen at whitespace
  breaklines=false,
  captionpos=b,                    % sets the caption-position to bottom
  commentstyle=\color{mygreen},    % comment style
  framexleftmargin=10pt,
  xleftmargin=10pt,
  framerule=0pt,
  frame=tb,                        % adds a frame around the code
  keepspaces=true,                 % keeps spaces in text, useful for keeping indentation of code (possibly needs columns=flexible)
  keywordstyle=\color{blue},       % keyword style
  showspaces=false,                % show spaces everywhere adding particular underscores; it overrides 'showstringspaces'
  showstringspaces=false,          % underline spaces within strings only
  showtabs=false,                  % show tabs within strings adding particular underscores
  stringstyle=\color{codestr},     % string literal style
  tabsize=2,                       % sets default tabsize to 2 spaces
}

\clearpage

\section{Problems}
\label{sec-1}

\subsection{Problem 1}
\label{sec-1-1}
Let $M$ be the set of all relations over $A=\{1, 2, 3\}$.
\begin{enumerate}
\item How many members does $M$ have?
\item Let $S$ be a set of relations over $M$, defined as follows:
\begin{equation*}
  S=\{R_1R_2 | R_1, R_2 \in M \land R_1R_2 = R_2R_1\}.
\end{equation*}

Show that $S$ is not an equivalence relation.
\end{enumerate}

\subsubsection{Answer 1}
\label{sec-1-1-1}
The number of elements of $M$ is $\exp(\left|A\right|}^2)/e^2 = 2^{3^2} = 2^9
    = 512$.  This can be proved from a partial sum of a recurence: $a_{n+1} =
    a_n + 2n - 1$, which describes the maximum number of ordered pairs possible
to create from $n$ elements and $b_n = 2b_{n-1}$ recurrence which describes
the number of elements of a powerset.  Since one can see that the number
of relations over a set is exactly the number of ways to subset the set of
all ordered pairs possible to create from that set, the final answer is
the composition of both recurrences: $c_n=b_n \circ a_n$.

Below I've implemented the counting algorithm in Prolog:

\lstset{language=prolog,numbers=none}
\begin{lstlisting}
powerset(Set, Result) :-
    powerset_helper(Set, [[]], Result).
powerset_helper([], X, X) :- !.
powerset_helper([X | Xs], In, Result) :-
    findall(Z, (member(Y, In), append([X], Y, Z)), Z),
    append(In, Z, Next),
    powerset_helper(Xs, Next, Result).

cross(Set, (A, B)) :- member(A, Set), member(B, Set).

pairs(Set, Pairs) :- findall(X, cross(Set, X), Pairs).

question_1(A) :-
    pairs(A, Pairs),
    powerset(Pairs, M),
    length(M, Len),
    format("$\\left|M\\right| = ~p$~n", [Len]).
\end{lstlisting}

$\left|M\right| = 512$
\subsubsection{Answer 3}
\label{sec-1-1-2}
I will prove that $S$ is not an equivalence using the definition of
equivalence which states that a relation is an equivalence if it is
\textbf{relfexive}, \textbf{symmetric} and \textbf{transitive}.  It is easy to see that the
definition of reflexivity requires that \emph{all} members of $M$ be present in
the relation, but, for example, $\{(1, 2)\}$ is absent from $S$.  Suppose,
for contradiction, it was present in $S$, then it would imply that there
exists a pair $\{(a, b)\} \in M$ s.t. $(\{(1, 2)\}, \{(a, b)\}) \in S$ and
$(\{(a, b)\}, \{(1, 2)\}) \in\nobreak S$.  By looking at the first and the
last members of the two members of $S$ we know that $\{(1, 2)\} \circ \{(a,
    b)\} = \{(1, b)\}$ and $\{(a, b)\} \circ \{(1, 2)\} = \{(a, 2)\}$, in other
words it is necessary that: $\{(1, 2)\} \circ \{(a, b)\} = \{(1, b)\} =
    \{(1, 2)\} = \{(a, 2)\} = \{(a, b)\} \circ \{(1, 2)\}$, which in turn, means
that $b=2$ and $a=1$.  Plugging these values back into original equilaty
gives $\{(1, 2)\} \circ \{(1, 2)\} = \{(1, 2)\}$, which is obviously false.
Hence, by contradiction, the proof is complete.
% Emacs 25.0.50.1 (Org mode 8.2.2)
\end{document}