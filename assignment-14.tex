% Created 2015-05-16 Sat 18:49
\documentclass[11pt]{article}
\usepackage[utf8]{inputenc}
\usepackage[T1]{fontenc}
\usepackage{fixltx2e}
\usepackage{graphicx}
\usepackage{longtable}
\usepackage{float}
\usepackage{wrapfig}
\usepackage{rotating}
\usepackage[normalem]{ulem}
\usepackage{amsmath}
\usepackage{textcomp}
\usepackage{marvosym}
\usepackage{wasysym}
\usepackage{amssymb}
\usepackage{capt-of}
\usepackage{hyperref}
\tolerance=1000
\usepackage[utf8]{inputenc}
\usepackage[usenames,dvipsnames]{color}
\usepackage{commath}
\usepackage{tikz}
\usetikzlibrary{shapes,backgrounds}
\usepackage{marginnote}
\usepackage{listings}
\usepackage{color}
\usepackage{enumerate}
\hypersetup{urlcolor=blue}
\hypersetup{colorlinks,urlcolor=blue}
\setlength{\parskip}{16pt plus 2pt minus 2pt}
\renewcommand{\arraystretch}{1.6}
\author{Oleg Sivokon}
\date{\textit{<2015-05-13 Wed>}}
\title{Assignment 14, Discrete Mathematics}
\hypersetup{
 pdfauthor={Oleg Sivokon},
 pdftitle={Assignment 14, Discrete Mathematics},
 pdfkeywords={Discrete Mathematics, Assignment, Relations},
 pdfsubject={Second asssignment in the course Discrete Mathematics},
 pdfcreator={Emacs 25.0.50.1 (Org mode 8.3beta)}, 
 pdflang={English}}
\begin{document}

\maketitle
\tableofcontents

\definecolor{codebg}{rgb}{0.96,0.99,0.8}
\definecolor{codestr}{rgb}{0.46,0.09,0.2}
\lstset{%
  backgroundcolor=\color{codebg},
  basicstyle=\ttfamily\scriptsize,
  breakatwhitespace=false,
  breaklines=false,
  captionpos=b,
  framexleftmargin=10pt,
  xleftmargin=10pt,
  framerule=0pt,
  frame=tb,
  keepspaces=true,
  keywordstyle=\color{blue},
  showspaces=false,
  showstringspaces=false,
  showtabs=false,
  stringstyle=\color{codestr},
  tabsize=2
}
\lstnewenvironment{maxima}{%
  \lstset{%
    backgroundcolor=\color{codebg},
    escapeinside={(*@}{@*)},
    aboveskip=20pt,
    captionpos=b,
    label=,
    caption=,
    showstringspaces=false,
    frame=single,
    framerule=0pt,
    basicstyle=\ttfamily\scriptsize,
    columns=fixed}}{}
}
\makeatletter
\newcommand{\verbatimfont}[1]{\renewcommand{\verbatim@font}{\ttfamily#1}}
\makeatother
\verbatimfont{\small}%
\clearpage

\section{Problems}
\label{sec:orgheadline10}

\subsection{Problem 1}
\label{sec:orgheadline3}
\begin{enumerate}
\item Develop the identity \((3 - 2)^n = 1\) using binom of Newton formula:
\begin{equation*}
  \sum_{i = 0}^n {?\choose ?} 3^? \cdot (??)^? = 1
\end{equation*}

And verify the identity for the case \(n = 4\).
\item Let number of ways to distribute \(k\) identical balls between 10
boxes is \(D(10, k)\).  Paint three boxes green and the remaining
seven---red.  Derive:
\begin{equation*}
  D(10, k) = \sum_{i = 0}^k ???,
\end{equation*}

and verify for the case \(k = 3\).
\end{enumerate}

\subsubsection{Answer 1}
\label{sec:orgheadline1}
\begin{equation*}
  \sum_{i = 0}^n {i\choose n} 3^i \cdot (-2)^{n - i} = 1.
\end{equation*}

\textbf{Solution:} \emph{(using Maxima)}
\begin{maxima}
n: 4;
tex(sum(binomial(n, i) * 3^i * (-2)^(n - i), i, 0, n));
(*@\label{orgsrcblock1}
@*)
\end{maxima}

\(1\)

\emph{(hand-made)}
\begin{align*}
  &\sum_{i = 0}^4 {i\choose 4} 3^i \cdot (-2)^{4 - i} \\
  &= 1 \cdot 3^0 \cdot (-2)^4 + 4 \cdot 3^1 \cdot (-2)^3 +
  6 \cdot 3^2 \cdot (-2)^2 + 4 \cdot 3^3 \cdot (-2)^1 +
  1 \cdot 3^4 \cdot (-2)^0 \\
  &= 16 - 96 + 216 - 216 + 81 \\
  &= 1
\end{align*}

\subsubsection{Answer 2}
\label{sec:orgheadline2}


\subsection{Prolbem 2}
\label{sec:orgheadline5}
How many permutations of a string \(AAABBCCDD\) can you form s.t. they
don't contain subsequences \(AAA\), \(BB\), \(CC\) or \(DD\)?

\subsubsection{Answer 3}
\label{sec:orgheadline4}
The total number of ways in which we can arrange the sequence \(AAABBCCDD\) is
\(n(\Omega)=\frac{9!}{3!2!2!2!}=7560\).  Then we find all permutations which
contain sequences of consequtive letters, \(AAA\), \(BB\) and so on.

\begin{align*}
  \frac{9!}{3!2!^3} &
  - \left(\frac{7!}{2!^3}       + 3\cdot\frac{8!}{3!2!^2}\right)
  + \left(3\cdot\frac{6!}{2!^2} + 3\cdot\frac{7!}{3!2!}\right)
  - \left(3\cdot\frac{5!}{2!}   + \frac{6!}{3!}\right)
  + 4! \\
  &= 7560 - (630 + 5040) + (540 + 1260) - (180 + 120) + 24 \\
  &= 3414\;.
\end{align*}

We count in stages: first we find all permutations of the string containing
\(AAA\) or duplicated characters, divided by the internal orderings of the
remaining duplicates.  Some of these permuations will also intersect with
each other, thus, we want to subtract duplicates such as \(AAA \cup BB\), but
now we subtracted some of the duplicates twice, so we need to add them back.
Those which we counted twice are those containing three subsequences, and so
on.  Finally:

\lstset{language=prolog,label= ,caption= ,captionpos=b,numbers=none}
\begin{lstlisting}
is_prefix([], _).
is_prefix(_, []) :- fail.
is_prefix([X | Xs], [X | Ys]) :- is_prefix(Xs, Ys).

not_allowed([a, a, a]).
not_allowed([b, b]).
not_allowed([c, c]).
not_allowed([d, d]).

prefix_allowed(Sofar) :-
    not_allowed(Bad), is_prefix(Bad, Sofar).

valid_seqence(X, [], X).
valid_seqence(Sofar, Pool, Result) :-
    select(E, Pool, Rem), Next = [E | Sofar],
    \+prefix_allowed(Next),
    valid_seqence(Next, Rem, Result).
valid_seqence(X) :-
    valid_seqence([], [a, a, a, b, b, c, c, d, d], X).

sans_repetitions :-
    findall(X, valid_seqence(X), X),
    list_to_set(X, Y),
    length(Y, Result),
    format('$$~p$$', [Result]).
\end{lstlisting}

\(3414\)

\subsection{Problem 3}
\label{sec:orgheadline7}
Four families (all distinct) went out to barbecue.  They took 8 steaks and 10
kebabs.  In how many ways is it possible to distribute the food to the
families, while every family has to have at least one meal?

\subsubsection{Answer 4}
\label{sec:orgheadline6}
We can distribute all meals in the following way: at first we will count
the total number of ways in which meals can be distributed, this is given by
\({10 + 4 - 1\choose 10}\cdot{8 + 4 - 1\choose 8}\).  Now, we need to subtract
the combinations where at least one family didn't get any food, add combinations,
where at least two families didn't get any food and subtract the combinations
where three familiies didn't get any food.
\begin{align*}
   & {4\choose 4} \cdot {10 + 4 - 1\choose 10} \cdot {8 + 4 - 1\choose 8} \\
  -& {4\choose 3} \cdot {10 + 3 - 1\choose 10} \cdot {8 + 3 - 1\choose 8} \\
  +& {4\choose 2} \cdot {10 + 2 - 1\choose 10} \cdot {8 + 2 - 1\choose 8} \\
  -& {4\choose 1} \cdot {10 + 1 - 1\choose 10} \cdot {8 + 1 - 1\choose 8} \\
  =& {} 286 * 165 - 4 * 66 * 45 + 6 * 11 * 9 - 4 \\
  =& {} 35900\;.
\end{align*}

\lstset{language=prolog,label= ,caption= ,captionpos=b,numbers=none}
\begin{lstlisting}
:- use_module(library(clpfd)).

feed_families([(S, K), (S1, K1), (S2, K2), (S3, K3)]) :-
    Steaks = [S, S1, S2, S3],
    Kebabs = [K, K1, K2, K3],
    Steaks ins 0..8, sum(Steaks, #=, 8),
    Kebabs ins 0..10, sum(Kebabs, #=, 10),
    append([Steaks, Kebabs], Meals),
    maplist(indomain, Meals),
    Total is (S + K) * (S1 + K1) * (S2 + K2) * (S3 + K3),
    Total > 0.

barbecue :-
    findall(X, feed_families(X), X),
    length(X, Result),
    format('$$~p$$', [Result]).
\end{lstlisting}

\(35900\)

\subsection{Problem 4}
\label{sec:orgheadline9}
Rami and Dina play a game where Dina selects 8 numbers in the \(10 \leq n \leq
   36\) range.  Rami has to find a way to form two sums from the numbers Dina
selected.

\subsubsection{Answer 5}
\label{sec:orgheadline8}
Dina can select seven even numbers (there are \((36 - 10) / 2 = 8\) such
numbers), and one odd number.  It wouldn't be possible to create two sums
from these numbers because adding odd numbers to even numbers must result in
an odd number, but since we have only one such number, we can't form an
equality involving that number.  Hence, an example of such set would be:
\(\{10, 12, 14, 16, 18, 20, 22, 23\}\).
\end{document}